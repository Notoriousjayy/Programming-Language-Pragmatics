\nopagenumbers
{\bf First-Class Values and Umlimited Extent}
\vskip 1mm
\hrule

\vskip 6pt
{\bf First-class}: A value that can be passed as a parameter, returned from a subroutine, or assigned into a variable.

\vskip 6pt
{\bf Second-class}: A value that can be passed as a parameter, but not returned from a subroutine or assigned into a variable.

\vskip 6pt
{\bf Third-class}: A value that cannot be passed as a parameter.

\vskip 6pt
{\bf Unlimited extent}: Objects lifetimes continues indefinitely.

\vskip 6pt
{\bf Limited extent}: Objects are destroyed at the end of their scope's execution.

\vskip 6pt
Simple types such as integers and characters are first-class values in most programming languages. If local objects were destroyed at the end of each scope's execution, then the referencing environment captured in a long-lived closure might become full of dangling references. To avoid this problem, most functional languages specify that local objects have {\it unlimited extent}. Their space can only be reclaimed when the garbage collection system is able to prove that they will never be used again. Local objects in most imperative languages have {\it limited extent}. Space for local objects with limited extent must generally be allocated on the stack. Space for local objects with unlimited extent must generally be allocated on a heap.

\vfill\eject
\bye
