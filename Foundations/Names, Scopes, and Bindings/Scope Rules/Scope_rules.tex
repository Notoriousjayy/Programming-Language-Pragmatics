\nopagenumbers
{\bf Scope Rules}
\vskip 1mm
\hrule

\vskip 6pt
{\bf Scope}: The textual region of the program in which a {\bf binding} is active. In most programming languages, the {\it scope} of a binding is determined {\bf statically} (at runtime). A scope is the body of a {\bf module, class, subroutine,} or {\bf structured control-flow} statement. A scope is sometimes called a {\bf block}.

\vskip 6pt
{\bf referencing environment}: The set of active {\bf bindings}. {\it referencing environment} is principally determined by static or dynamic {\bf scope rules}. A referencing environment generally corresponds to a sequence of scopes that can be examined (in order) to find the current binding for a given name.

\vskip 6pt
In some case, referencing environments depends on {\bf binding rules}. Specifically, when a reference to a subroutine $S$ is stored in a variable, passed as a parameter to another subroutine, or returned as a function value, one needs to determine when the referencing environment for $S$ is chosen--- that is, when the binding between the reference to $S$ and the referencing environment of $S$ is made.

\vskip 6pt
\centerline{$\underline{\hbox{{\bf Two Principle Binding Rule Options}}}$}

\vskip 6pt
{\bf Deep binding}: The choice is made when the reference is first created.

\vskip 6pt
{\bf Shallow binding}: the choice is made when the {\bf reference} is finally used.

\vfill\eject
