\centerline{\bf Building a Runnable Program}

\vskip 1cm

1. What is a code generator? Why might it be useful?

\filbreak
\vskip 1cm

2. What is a basic block? A control flow graph?

\filbreak
\vskip 1cm

3. What are virtual registers? What purpose do they serve?

\filbreak
\vskip 1cm

4. What is the difference between local and global code improvement?

\filbreak
\vskip 1cm

5. What is register spilling?

\filbreak
\vskip 1cm

6. Explain what is meant by the "level" of an intermediate form (IF). What are the comparative advantages and disadvantages of high-, medium-, and low-level IFs?

\filbreak
\vskip 1cm

7. What is the IF most commonly used in Ada compilers?

\filbreak
\vskip 1cm

8. Name two advantages of a stack-based IF. Name one disadvantage.

\filbreak
\vskip 1cm

9. Explain the rationale for basing a family of compilers (several languages, several target machines) on a single IF.

\filbreak
\vskip 1cm

10. Why might a compiler employ more than one IF?

\filbreak
\vskip 1cm

11. Outline some of the major design alternatives  for back-end compilers organizaion and structure.

\filbreak
\vskip 1cm

12. What is sometimes called the "middle end" of a compiler?

\filbreak
\vskip 1cm

13. Why is management of a limited set of physical registers usually deferred until late in the compilation process?

\filbreak
\vskip 1cm

14. What are the distinguishing characteristics of a relocatable object file? An executable object file?

\filbreak
\vskip 1cm

15. Why do operating systems typically zero-fill pages used for unitinitalized data?

\filbreak
\vskip 1cm

16. List four tasks commonly performed by an assembler.

\filbreak
\vskip 1cm

17. Summarize the comparative advantages of assembly languages and object code as the output of a compiler.

\filbreak
\vskip 1cm

18. Give three examples of pseudoinstructions and three examples of directives that an assembler might be likely to provide.

\filbreak
\vskip 1cm

19. Why might an assembler perform its own final pass of instruction scheduling?

\filbreak
\vskip 1cm

20. Explain the distinction between absolute and relocatable words in an object file. Why is the notion of "relocatability" more complicated than it used to be?

\filbreak
\vskip 1cm

21. What is the difference between linking and loading?

\filbreak
\vskip 1cm

22. What are the principal tasks of a linker?

\filbreak
\vskip 1cm

23. How can a linker enforce type checking across compilation units?

\filbreak
\vskip 1cm

24. What is the motivation for dynamic linking?

\filbreak
\vskip 1cm

25. Characterize GIMPLE, RTL, Java bytecode, and Common Intermediate Language as high-, medium, or low-level intermediate forms.

\filbreak
\vskip 1cm

26. Nae three languages (other than C) for which there exist {\tt gcc} front ends.

\filbreak
\vskip 1cm

27. What is the internal IF of {\tt gcc}'s front ends?

\filbreak
\vskip 1cm

28. Give brief descriptions of GIMPLE and RTL. How do they differ? Why was GIMPLE introduced?

\filbreak
\vskip 1cm

29. Explain the addressing challenge faced by dynamic linking systems.

\filbreak
\vskip 1cm

30. What is position-independent code? What is it good for? What special precautions must a compiler follow in order to produce it?

\filbreak
\vskip 1cm

31. Explain the need for PC-relative addressing in position-independent code. How is it accomplished on the x86-32?

\filbreak
\vskip 1cm

32. What is the purpose of a linkage table?

\filbreak
\vskip 1cm

33. What is lazy dynamic linking? What is its purpose? How does it work?

\filbreak
\vfill\eject
\bye
