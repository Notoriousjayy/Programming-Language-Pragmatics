\centerline{\bf Subroutines and Control Abstraction}

\vskip 1cm

1. What is a subroutine calling sequence? What does it do? What is meant by the subroutine prologue and epilogue?

\filbreak
\vskip 1cm

2. How do calling sequences typically differ in older ({\tt CISC}) and newer ({\tt RISC}) instruction sets?

\filbreak
\vskip 1cm

3. Describe how to maintain the static chain during a subroutine call.

\filbreak
\vskip 1cm

4. What is a display? How does it differ from a static chain?

\filbreak
\vskip 1cm

5. What are the purposes of the stack pointer and frame pointer registers? Why does a subroutine often need both?

\filbreak
\vskip 1cm

6. Why do modern machines typically subroutine parameters in registers rather than on the stack?

\filbreak
\vskip 1cm

7. Why do subroutine calling conventions often give the caller responsibility for saving half the registers and the callee responsible for saving the other half?

\filbreak
\vskip 1cm

8. If work can be done in either the caller or the callee, why do we typically prefer to do it in the callee?

\filbreak
\vskip 1cm

9. Why do compilers typically allocate space for arguments in the stack, even when they pass them in registers?

\filbreak
\vskip 1cm

10. List the optimizations that can be made to the subroutine calling sequence in important special cases (e.g., leaf routines).

\filbreak
\vskip 1cm

11. How does an in-line subroutine differ from a macro?

\filbreak
\vskip 1cm

12. Under what circumstances is it desirable to expand a subroutine in-line?

\filbreak
\vskip 1cm

13. What is the difference between formal and actual parameters?

\filbreak
\vskip 1cm

14. Describe four common parameter-passing modes. How does a programmer choose which one to use when?

\filbreak
\vskip 1cm

15. Explain the rationale for {\tt READONLY} parameters in Modula-3.

\filbreak
\vskip 1cm

16. What parameter mode is typically used in languages with a reference model of variables?

\filbreak
\vskip 1cm

17. Describe the parameter modes of Ada. How do they differ from the modes of other modern languages?

\filbreak
\vskip 1cm

18. Give an example in which it is useful to return a reference from a function in C++.

\filbreak
\vskip 1cm

19. What is an r-value reference? Why might it be useful?

\filbreak
\vskip 1cm

20. List three reasons why a language implementation might implement a parameter as a closure.

\filbreak
\vskip 1cm

21. What is a conformant (open) array?

\filbreak
\vskip 1cm

22. What are default parameters? Why are they useful?

\filbreak
\vskip 1cm

23. What are named (keyword) parameters? Why are they useful?

\filbreak
\vskip 1cm

24. Explain the value of variable-length argument lists. What distinguishes such lists in Java and C$\#$ from their counterparts in C and C++?

\filbreak
\vskip 1cm

25. Describe three common mechanisms for specifying the return value of a function. What are their relative strengths and drawbacks?

\filbreak
\vskip 1cm

26. Describe three ways in which a language may allow programmers to declare exceptions.

\filbreak
\vskip 1cm

27. Explain why it is useful to define exceptions as classes in C++, Java, and C$\#$.

\filbreak
\vskip 1cm

28. Explain te behavior and purpose of a {\tt try...finally} construct.

\filbreak
\vskip 1cm

29. Describe the algorithm used to identify an appropriate handler when an exception is raised in a language like Ada or C++.

\filbreak
\vskip 1cm

30. Explain how to implement exceptions in a way that incurs no cost in the common case (when exceptions don't arise).

\filbreak
\vskip 1cm

31. How do the exception handlers of a functional language like ML differ from those of an imperative language like C++?

\filbreak
\vskip 1cm

32. Describe the operations that must be performed by the implicit handler for a subroutine.

\filbreak
\vskip 1cm

33. Summarize the shortcomings of the {\tt setjump} and {\tt longjump} library routines of C.

\filbreak
\vskip 1cm

34. What is a {\tt volatile} variable in C? Under what circumstances is it useful?

\filbreak
\vskip 1cm

35. What was the first high-level programming language to provide coroutines?

\filbreak
\vskip 1cm

36. What is the difference between a coroutine and a thread?

\filbreak
\vskip 1cm

37. Why doesn't the transfer library routine need to change the program counter when switching between coroutines?

\filbreak
\vskip 1cm

38. Describe three alternative means of allocating coroutine stacks. What are their relative strengths and weakneses?

\filbreak
\vskip 1cm

39. What is a cactus stack? What is its purpose?

\filbreak
\vskip 1cm

40. What is discrete even simulation? What is its connection with coroutines?

\filbreak
\vskip 1cm

41. What is an event in the programming language sense of the word?

\filbreak
\vskip 1cm

42. Summarize the two main implementation strategies for events.

\filbreak
\vskip 1cm

43. Explain the appeal of anonymous delegats (C$\#$) and anonymous inner classes (Java) for handling events.

\filbreak
\vskip 1cm

44. Describe how we access an object at lexical nesting level $k$ in a language implementation based on displays.

\filbreak
\vskip 1cm

45. Why isn't the display typically kept in registers?

\filbreak
\vskip 1cm

46. Explain how to maintain the display during subroutine calls?
\filbreak
\vskip 1cm

47. What special concerns arise when creating closures in a language implementation that uses displays?

\filbreak
\vskip 1cm

48. Summarize the tradeoffs between displays and static chains. Describe a program for which displays will result in faster code. Describe another for which static chains will be faster.

\filbreak
\vskip 1cm

49. For each of our three case studeis, explain which aspects of the calling sequence and stack layout are dictated by the hardware, and which are a matter of software convention.

\filbreak
\vskip 1cm

50. Why don't {\tt LLVM} and {\tt gcc} restore caller-saves registers immediately after a call?

\filbreak
\vskip 1cm

51. What is a subroutine closure trampoline? How does it differ from the usual implementation of a closure described in Section 3.6.1? What are the comparative advantages of the two alternatives?

\filbreak
\vskip 1cm

52. Explain the circumstances under which a subroutine needs a frame pointer (i.e., under which access via displacement addressing from the stack pointer will not suffice).

\filbreak
\vskip 1cm

53. Under what circumstances must an argument that was passed in a register also be saved into the stack?

\filbreak
\vskip 1cm

54. What is the purpose of the "red zone" on x86-64?

\filbreak
\vskip 1cm

55. What are register windows? What purpose do they serve?

\filbreak
\vskip 1cm

56. Which commercial instruction sets include regiser windows?

\filbreak
\vskip 1cm

57. Explain the concepts of register window overflow and underflow.

\filbreak
\vskip 1cm

58. Why are register windows a potentional problem for multithreaded programs?

\filbreak
\vskip 1cm

59. What is call by name? What language first provided it? Why isn't it used by the language's descendants?

\filbreak
\vskip 1cm

60. What is call by need? How does it differ from call by name? What modern languages use it?

\filbreak
\vskip 1cm

61. How does a subroutine with call-by-name parameters differ from a macro?

\filbreak
\vskip 1cm

62. What is a thunk? What is it used for?

\filbreak
\vskip 1cm

63. What is Jensen's device?

\filbreak
\vskip 1cm

64. Describe the "obvious" implementation of iterators using coroutines.

\filbreak
\vskip 1cm

65. Explain how the state of multiple active iterators can be maintained in a single stack.

\filbreak
\vskip 1cm

66. Describe the transformation used by C$\#$ compilers to turn a true iterator into an iterator object.

\filbreak
\vskip 1cm

67. Summarize the computational model of discrete event simulation. Explain the significance of the time-based priority queue.

\filbreak
\vskip 1cm

68. When building a discrete event simulation, how does one decide which things to model with coroutines, and which to model with data structures?

\filbreak
\vskip 1cm

69. Are all inactive coroutines huaranteed to be in the priority queue? Explain.

\filbreak
\vfill\eject
\bye
