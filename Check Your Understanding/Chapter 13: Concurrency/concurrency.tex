\centerline{\bf Concurrency}

\vskip 1cm

1. Explain the distinctions among concurrent, parallel, and distributed.

\filbreak
\vskip 1cm

2. Explain the motivation for concurrency. Why do people write concurrent programs? What accounts for the increased interest in concurrency in recent years?

\filbreak
\vskip 1cm

3. Describe the implementation levels at which parallelism appears in modern systems, and the levels of abstraction at which it may be considered by the programmer.

\filbreak
\vskip 1cm

4. What is a race condition? What is synchronization?

\filbreak
\vskip 1cm

5. What is a context switch? Preemption?

\filbreak
\vskip 1cm

6. Explain the concept of a dispatch loop. What are its advantages and disadvantages with resepct to multithreaded code?

\filbreak
\vskip 1cm

7. Explain the distinction between a multiprocessor and a cluster; between a processor and a core.

\filbreak
\vskip 1cm

8. What does it mean for memory in a multiprocessor to be uniform? What is the alternative?

\filbreak
\vskip 1cm

9. Explain the coherence problem for multicore and multiprocessor caches.

\filbreak
\vskip 1cm

10. What is a vector machine? Where does vector technology appear in modern systems?

\filbreak
\vskip 1cm

11. Explain the differences among coroutines, threads, lightweight processes, and heavyweight processes.

\filbreak
\vskip 1cm

12. What is a quasiparallelism?

\filbreak
\vskip 1cm

13. Describe the bag of tricks programming model.

\filbreak
\vskip 1cm

14. What is busy-waiting? What is its principal alternative?

\filbreak
\vskip 1cm

15. Name four explicitly concurrent programming languages.

\filbreak
\vskip 1cm

16. Why don't message-passing programs require explicit synchronization mechanisms.

\filbreak
\vskip 1cm

17. What are the tradeoffs between language-based and library-based implementations of concurrency?

\filbreak
\vskip 1cm

18. Explain the difference between data parallelism and task parallelism.

\filbreak
\vskip 1cm

19. Describe six different mechanisms commonly used to create new threads of control in a concurrent program.

\filbreak
\vskip 1cm

20. In what sense is {\tt fork/join} more powerful than co-begin?

\filbreak
\vskip 1cm

21. What is a thread pool in Java? What purpose does it serve?

\filbreak
\vskip 1cm

22. What is meant by a two-levelthread implementation?

\filbreak
\vskip 1cm

23. What is a ready list?

\filbreak
\vskip 1cm

24. Describe the progressive implementation of scheduling, preemption, and (true) parallelism on top of coroutines.

\filbreak
\vskip 1cm

25. What is mutual exclusion? What is a critical section?

\filbreak
\vskip 1cm

26. What does it mean for an operation to be atomic? Explain the difference between atomicity and condition synchronization.

\filbreak
\vskip 1cm

27. Describe the behavior of a {\tt test$\_$and$\_$set} instruction. Show how to use it to build a spin lock.

\filbreak
\vskip 1cm

28. Describe the behavior of the {\tt compare$\_$and$\_$swap} instructions. What advantages does it offer in comparison to {\tt test$\_$and$\_$set}?

\filbreak
\vskip 1cm

29. Explain how a reader-writer lock differs from an "ordinary" lock.

\filbreak
\vskip 1cm

30. What is a barrier? In what types of programs are barriers common?

\filbreak
\vskip 1cm

31. What does it mean for an algorithm to be nonblocking? What advantages do nonblocking algorithms have over algorithms based on locks?

\filbreak
\vskip 1cm

32. What is sequential consistency? Why is it difficult to implement?

\filbreak
\vskip 1cm

33. What information is provided by a memory consistency model? What is the relationship between hardware-level and language-level memory models.

\filbreak
\vskip 1cm

34.  Explain how to extend a preemptive uniprocessor scheduler to work correctly on a multiprocessor.

\filbreak
\vskip 1cm

35. What is a spin-then-yield lock?

\filbreak
\vskip 1cm

36. What is a bounded buffer?

\filbreak
\vskip 1cm

37. What is a semaphore? What operations does it support? How do binary and general semaphores differ?

\filbreak
\vskip 1cm

38. What is a monitor? How do monitor condition variables differ fro semaphores?

\filbreak
\vskip 1cm

39. Explain the difference between treating treating monitor signals as hints and treating them as absolutes.

\filbreak
\vskip 1cm

40. What is a monitor invariant? Under what circumstances must it be guaranteed to hold?

\filbreak
\vskip 1cm

41. Describe the nested monitor problem and some potential solutions.

\filbreak
\vskip 1cm

42. What is deadlock?

\filbreak
\vskip 1cm

43. What is a conditional region? How does it differ from a monitor?

\filbreak
\vskip 1cm

44. Summarize the synchronization mechanisms of Ada 95, Java, and C$\#$. Contrast them with one another, and with monitors and conditional critical regions. Be sure to explain the features added to Java 5.

\filbreak
\vskip 1cm

45. What is transactional memory? What advantages does it offer over algorithms based on locks? What challenges will need to be overcome before it enters widespread use?

\filbreak
\vskip 1cm

46. Describe the semantics of the HPF/Fortran 95 {\tt forall} loop. How does it differ from {\tt do concurrent}?

\filbreak
\vskip 1cm

47. Why might pure functional languages be said to provide a particularly attractive setting for concurrent programming?

\filbreak
\vskip 1cm

48. What are futures? In what languages do they appear? What precautions must the programmer take when using them?

\filbreak
\vskip 1cm

49. Explain the difference between {\tt AND} parrallelism and {\tt OR} parallelism in Prolog.

\filbreak
\vskip 1cm

50. Describe three ways in which processes or threads commonly name their communication partners.

\filbreak
\vskip 1cm

51. What is a datagram?

\filbreak
\vskip 1cm

52. Why, in general, might a {\tt send} operation need to block?

\filbreak
\vskip 1cm

53. What are the three principal synchronization options for the sender of a message? What are the tradeoffs among them?

\filbreak
\vskip 1cm

54. What are {\tt gather} and {\tt scatter} operations in a message-passing program? What are marshalling and unmarshalling?

\filbreak
\vskip 1cm

55. Describe the tradeoffs between explicit and implicit message receipt.

\filbreak
\vskip 1cm

56. What is a remote procedure call (RPC)? What is a stub compiler?

\filbreak
\vskip 1cm

57. What are the obstacles to transparency in an RPC system?

\filbreak
\vskip 1cm

58. What is a rendezous? How does it differ from a remote procedure call?

\filbreak
\vskip 1cm

59. Explain the purpose of a {\tt select} statement in Ada or Go.

\filbreak
\vskip 1cm
60. What semantic and pragmatic challenges are introduced by the ability to "peek" inside messages before they are received?

\filbreak
\vfill\eject
\bye
