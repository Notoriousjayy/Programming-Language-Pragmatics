\centerline{\bf Semantic Analysis}

\vskip 1cm

1. What determines whether a language rule is a matter of syntax or of static semantics?

\filbreak
\vskip 1cm

2. Why is it impossible to detect certain program errors at compile time, even though they can be detected at run time?

\filbreak
\vskip 1cm

3. What is an attribute grammar?

\filbreak
\vskip 1cm

4. What are programming assertions? What is their purpose?

\filbreak
\vskip 1cm

5. What is the difference between synthesized and inherited attributes?

\filbreak
\vskip 1cm

6. Give two examples of information that is typically passed through inherited attributes.

\filbreak
\vskip 1cm

7. What is attribute flow?

\filbreak
\vskip 1cm

8. What is a one-pass compiler?

\filbreak
\vskip 1cm

9. What does it mean for an trribute grammar to be S-attributed? L-attributed? Noncircular? What is the significance of these grammar classes?

\filbreak
\vskip 1cm

10. What is the difference between a semantic function and an action routine?

\filbreak
\vskip 1cm

11. Why can't action routines be placed at arbirary locations within the right-hand side of productions in an {\tt  LR} CFG?

\filbreak
\vskip 1cm

12. What patterns of attribute flow can be captured easily with action routines?

\filbreak
\vskip 1cm

13. Some compilers perform all semantic checks and intermediate code generation in action routines. Others use action routines to build a syntax tree and then perform checks and intermediate code generation in separate traversals of the syntax tree. Discuss the tradeoffs between these two strategies.

\filbreak
\vskip 1cm

14. What sort of information do action routines typically keep in global variables, rather than in attributes?

\filbreak
\vskip 1cm

15. Describe the similarites and differences between context-free grammars and tree grammars.

\filbreak
\vskip 1cm

16. How can a semantic analyzer avoid the generation of cascading errors messages?

\filbreak
\vskip 1cm

17. Explain how to manage space for synthesized attributes in a bottom-up parser.

\filbreak
\vskip 1cm

18. Explain how to manage space for inerited attributes in a bottom-up parser.

\filbreak
\vskip 1cm

19. Define left corner and trailing part.

\filbreak
vskip 1cm

20. Under what circumstances can an action routine be embedded in the right-hand side of a production in a bottom-up parser? Equivalently, under what circumstanes can a marker symbol be embedded in a right-hand side without rendering the grammar non-{\tt LR}?

\filbreak
\vskip 1cm

21. Summarize the tradeoffs between the ad hoc management of space for attributes in a top-down parser.

\filbreak
\vskip 1cm

22. At any given point in a top-down parse, which symbols wil have attribute records in an automatically managed attribute stack?

\filbreak
\vskip 1cm
\filbreak
\vfill\eject
\bye
