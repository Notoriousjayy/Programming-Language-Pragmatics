\centerline{\bf Scripting Languages}

\vskip 1cm

1. Give a one sentence definition of "scripting language."

\filbreak
\vskip 1cm

2. List the principal ways in which scripting languages differ from conventional "systems" languages.

\filbreak
\vskip 1cm

3. From what two principal sets of ancestors are modern scriping languages descended?

\filbreak
\vskip 1cm

4. What IBM creation is generally considered the first general-purpose scripting language?

\filbreak
\vskip 1cm

5. What is the most popular language for server-side web scripting?

\filbreak
\vskip 1cm

6. How does the notion of context in Perl differ from coercion?

\filbreak
\vskip 1cm

7. What is globbing? What is a wildcard?

\filbreak
\vskip 1cm

8. What is a pipe in Unix? What is redirection?

\filbreak
\vskip 1cm

9. Describe the three standard {\tt I/O} streams provided to every Unix process.

\filbreak
\vskip 1cm

10. Explain the significance of the $\# !$ convention in Unix shell scripts.

\filbreak
\vskip 1cm

11. What is meant by the pattern space in {\tt sed}?

\filbreak
\vskip 1cm

12. Briefly describe the fields and associative arrays of {\tt awk}.

\filbreak
\vskip 1cm

13. In what ways even early versions of Perl improve on {\tt sed} and {\tt awk}?

\filbreak
\vskip 1cm

14. Explain the special relationship between {\tt while} loops and file handlers in Perl. What is the meaning of the empty file handle, {\tt <>}?

\filbreak
\vskip 1cm

15. Name three widely used commercial packages for mathematical computing.

\filbreak
\vskip 1cm

16. List several distinctive features of the R statistical scripting language.

\filbreak
\vskip 1cm

17. Explain the meaning of the $\$$ and @ characters at the beginning of variable names in Perl. Explain the different meanings of $\$$, @, and @@ in Ruby.

\filbreak
\vskip 1cm

18. Which of the languages described in Section 12.2.4 uses indentation to control syntactic grouping?

\filbreak
\vskip 1cm

19. List several distinctive features of Python.

\filbreak
\vskip 1cm

20.Describe, briefly, how Ruby uses blocks and iterators.

\filbreak
\vskip 1cm

21. What capabilities must a scripting language provide in order to be used for extension?

\filbreak
\vskip 1cm

22. Name several commercial tools that use extension languages.

\filbreak
\vskip 1cm

23. Explain the distinction between server-side and client-side web scripting.

\filbreak
\vskip 1cm

24. List the tradeoffs between CGI scripts and embedded PHP.

\filbreak
\vskip 1cm

25. Why are CGI sripts usually installed only in a special directory?

\filbreak
\vskip 1cm

26. Explain how a PHP page can service its own requests.

\filbreak
\vskip 1cm

27. Why might we prefer to execute a web script on the server rather than the client? Why might we sometimes prefer the client instead?

\filbreak
\vskip 1cm

28. What is the HTML Document Object Model? What is its significance for client side scripting?

\filbreak
\vskip 1cm

29. What is the relationship between JavaScript and Java?

\filbreak
\vskip 1cm

30. What is an applet? Why applets are usually not considered an example of scripting?

\filbreak
\vskip 1cm

31. What is HTML? XML? XSLT? How are they related to one another?

\filbreak
\vskip 1cm

32. Name a scripting language that uses dynamic scoping.

\filbreak
\vskip 1cm

33. Summarize the strategies used in Perl, PHP, Ruby, and Python to determine the scope of variables that are not declared.

\filbreak
\vskip 1cm

34. Describe the conceptual model for dynamically scoped variables in Perl.

\filbreak
\vskip 1cm

35. List the principal features found in POSIX regular expressions, but not in the regular expressions of formal language theory (Section 2.1.1).

\filbreak
\vskip 1cm

36. List the principal features found in Perl REs, but not in those of POSIX.

\filbreak
\vskip 1cm

37. Explain the purpose of search modifiers (characters following the final delimiter) in Perl-type regular expressions.

\filbreak
\vskip 1cm

38. Describe the three main categories of escape sequences in Perl-type regular expressions.

\filbreak
\vskip 1cm

39. Explain the difference between greedy and minimal matches.

\filbreak
\vskip 1cm

40. Describe the notion of capture in regular expressions.

\filbreak
\vskip 1cm

41. Contrast the philosophies of Perl and Ruby with regard to error checking and reporting.

\filbreak
\vskip 1cm

42. Compare the numeric types of popular scripting languages to those of compiled languages like C or Fortran.

\filbreak
\vskip 1cm

43. What are bignums? Which languages support them?

\filbreak
\vskip 1cm

44. What are associative arrays? By what other names are they sometimes known?

\filbreak
\vskip 1cm

45. Why don't most scripting languages provide direct support for records?

\filbreak
\vskip 1cm

46. What is a typeglob in Perl? What purpose does it serve?

\filbreak
\vskip 1cm

47. Describe the tuple and set types of Python.

\filbreak
\vskip 1cm

48. Explain the unification of arrays and hashs in PHP and Tcl.

\filbreak
\vskip 1cm

49. Explain the unification of arrays and objects in JavaScript.

\filbreak
\vskip 1cm

50. Explain how tuples and hashes can be used to emulate multidimensional arrays in Python.

\filbreak
\vskip 1cm

51. Explain the concept of context in Perl. How is it related to type compatibility and type inference? What are the two principal contex defined by the language's operators?

\filbreak
\vskip 1cm

52. Compare the approaches to object orientation taken by Perl 5, PHP 5, JavaScript, Python, and Ruby.

\filbreak
\vskip 1cm

53. What is meant by the blessing of a reference in Perl?

\filbreak
\vskip 1cm

54. What are prototypes in JavaScript? What purpose do they serve?

\filbreak
\vskip 1cm

55. Explain the relationship among SGML, HTML, and XML. What are their corresponding stylesheet languages?

\filbreak
\vskip 1cm

56. Why does XML work so hard to distinguish between content and presentation?

\filbreak
\vskip 1cm

57. What are the four main components of XSL? What are their respective purposes?

\filbreak
\vskip 1cm

58. What is XHTML? How does it differ from "ordinary" HTML?

\filbreak
\vskip 1cm

59. Explain the correspondence between XML documents and trees.

\filbreak
\vskip 1cm

60. What does it mean for an XML document to be well formed.

\filbreak
\vskip 1cm

61. Explain the distinctions (syntactic and semantic) among elements, declarations, and processing instructions in XML. Also explain the distinctions among elements, tags, and attributes.

\filbreak
\vskip 1cm

62. Summarize the execution model of XSLT. In a nutshell, how does it work?

\filbreak
\vskip 1cm

63. Explain the difference between applying templates and calling them in XSLT.

\filbreak
\vfill\eject
\bye
