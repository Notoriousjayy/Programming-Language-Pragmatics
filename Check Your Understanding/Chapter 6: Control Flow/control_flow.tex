\centerline{\bf Cotrol Flow}

\vskip 1cm

1. Nane eight major categories of control-flow mechanisms.

\filbreak
\vskip 1cm

2. What distinguishes operators from other sorts of functions?

\filbreak
\vskip 1cm

3. Explain the difference between prefix, infix, and postfix notation. What is Cambridge Polisg notation? Name two programming languages that use postfix notation.

\filbreak
\vskip 1cm

4. Why don't issues of associativity and precedence arise in Postscript or Forth?

\filbreak
\vskip 1cm

5. What does it mean for an expression to be referentially transparent?

\filbreak
\vskip 1cm

6. What is the difference between a value mode of variables and a reference model of variables? Why is the distinction important?

\filbreak
\vskip 1cm

7. What is an l-value? An r-value?

\filbreak
\vskip 1cm

8. Why is the distinction between mutable and immutable values important in the implementation of a language with a reference model of variables?

\filbreak
\vskip 1cm

9. Define orthogonaality in the context of a programming language design.

\filbreak
\vskip 1cm

10. What is the difference between a statement and an expression? What does it mean for a language to be expression-oriented?

\filbreak
\vskip 1cm

11. What are the advantages of updating a variable with an assignment operator, rather than with a regular assignment in which the variable appears on both the left-hand side and right-hand side?

\filbreak
\vskip 1cm

12. Given the ability to assign a value into a variable, wy is it useful to be able to specify an initial value?

\filbreak
\vskip 1cm

13. What are aggregates? Why are they useful?

\filbreak
\vskip 1cm

14. Explain the notion of definite assignment in Java and C$\#$.

\filbreak
\vskip 1cm

15. Why is it generally expensive to catch all uses of uninitialized variables at run time?

\filbreak
\vskip 1cm

16. Why is it impossible to catch all uninitialized variables at compile time?

\filbreak
\vskip 1cm

17. Why do most languages leave unspecified the order in which the arguments of an operator or function are evaluated?

\filbreak
\vskip 1cm

18. What is short-circuit Boolean evaluation? Why is it useful?

\filbreak
\vskip 1cm

19. List the principal uses of {\tt goto}, and the structured alternatives to each.

\filbreak
\vskip 1cm

20. Explain the distinction between exceptions and multilevel returns.

\filbreak
\vskip 1cm

21.What are continuations? What other language features do they subsume?

\filbreak
\vskip 1cm

22.  Why is sequencing a comparatively unimportant form of control flow in Lisp?

\filbreak
\vskip 1cm

23. Explain why it may sometimes be useful for a function to have side effects.

\filbreak
\vskip 1cm

24. Describe the {\tt jump} code implementation of short-circuit Boolean evaluation.

\filbreak
\vskip 1cm

25. Why do imperative languages commonly provide a {\tt case} or {\tt switch} statement in addition to {\tt if...then...else}?

\filbreak
\vskip 1cm

26. Describe three different search strategies that might be employed in the implementation of a {\tt case} statement, and the circumstances in which each would be desirable.

\filbreak
\vskip 1cm

27. Explain the use of {\tt break} to terminate the arms of a C {\tt switch} statement, and the behavior that arises if a {\tt break} is accidentally omitted.

\filbreak
\vskip 1cm

28. Describe three subtleties in the implementation of enumeration-controlled loops.

\filbreak
\vskip 1cm

29. Why do most languages not allow the bounds of increment of an enumeration-controlled loop to be floating-point numbers?

\filbreak
\vskip 1cm

30. Why do many languages require the step size of an enumeration-controlled loop to be a compile-time constant?

\filbreak
\vskip 1cm

31. Describe the "iteration count" loop implementation. What problem(s) does it solve?

\filbreak
\vskip 1cm

32. What are the advantages of making an index variable local to the loop it controls?

\filbreak
\vskip 1cm

33. Does C have enumerations-controlled loops? Explain.

\filbreak
\vskip 1cm

34. What is a collection ( a container instance)?

\filbreak
\vskip 1cm

35. Explain the difference between true iterators and iterator objects.

\filbreak
\vskip 1cm

36. Cite two advantages of iterator objects over the use of programming conventions in a language like C.

\filbreak
\vskip 1cm

37. Describe the approach to iteration typically employed in languages with first-class functions.

\filbreak
\vskip 1cm

38. Given an example in which a mid-test loop results in more elegant code than does a pretest or post-test loop.

\filbreak
\vskip 1cm

39. What is a tail-recursive function? Why is tail recursion important?

\filbreak
\vskip 1cm

40. Explain the difference between applicative-order and normal-order evaluation of expressions. Under what circumstances is each desirable?

\filbreak
\vskip 1cm

41. What is lazy evaluation? What are promises? What is memoization?

\filbreak
\vskip 1cm

42. Give two reasons why lazy evaluation may be desirable.

\filbreak
\vskip 1cm

43. Name a language in which parameters are always evaluated lazily.

\filbreak
\vskip 1cm

44. give two reasons why a prorammer might sometimes want control flow to be nondeterministic.

\filbreak
\vskip 1cm

45. Explain how Icon generators differ from the iterators of Clu, Python, Ruby, and C$\#$, and from the iterator objects of Euclid, C++, and Java.

\filbreak
\vskip 1cm

46. Describe the notations of success and failure in Icon.

\filbreak
\vskip 1cm

47. What is backtracking? Why is it useful?

\filbreak
\vskip 1cm

48. Name a language other than Icon n which backtracking plays a fundamental role.

\filbreak
\vskip 1cm

49. What is a guarded command?

\filbreak
\vskip 1cm

50. Explain why nondeterminacy is particularly important for concurrent programs.

\filbreak
\vskip 1cm

51. Give three alternative definitions of fairness in the context of nondeterminacy.

\filbreak
\vskip 1cm

52. Desrie three possible ways of implementing the choice among guards that evaluate to true. What are the tradeoffs among these?

\filbreak
\vfill\eject
\bye
