\centerline{\bf Run-Time Program Maagement}

\vskip 1cm

1. What is a run-time system? How does it differ from a "mere" library?

\filbreak
\vskip 1cm

2. List some of the major tasks that may be performed by a run-time system.

\filbreak
\vskip 1cm

3. What is a virtual machine? What distingushes it from interpreters of other sorts?

\filbreak
\vskip 1cm

4. Explain the distinction between system and process VMs. What other terms are sometimes used for system VMs?

\filbreak
\vskip 1cm

5. What is managed code?

\filbreak
\vskip 1cm

6. Why do many virtual machines use a stack-based intermediate form?

\filbreak
\vskip 1cm

7. Give serveral examples of special purpose instructions provided by Java bytecode/

\filbreak
\vskip 1cm

8. Summarize the architecture of the Java Virtual Machine.

\filbreak
\vskip 1cm

9. Summarize the content of a Java class file.

\filbreak
\vskip 1cm

10. Explain the validity checks performed on a class file at load time.

\filbreak
\vskip 1cm

11. What is a just-in-time (JIT) compiler? What are its potential advantages over interpretation or conventional compilation?

\filbreak
\vskip 1cm

12. Why might one prefer bytecode over source code as the input to a JIT compiler?

\filbreak
\vskip 1cm

13. What distinguishes dynamic compilation from just-in-time compilation?

\filbreak
\vskip 1cm

14. What is a hot path? Why is it significant?

\filbreak
\vskip 1cm

15. Under what circumstances can a JIT compiler expand virtual methods in-line?

\filbreak
\vskip 1cm

16. What is deoptimization? When and why is it needed?

\filbreak
\vskip 1cm

17. Explain the distinction between the function and expression tree representations of a lambda expression in C$\#$.

\filbreak
\vskip 1cm

18. Summarize the relationship between compilation and interpretation in Perl.

\filbreak
\vskip 1cm

19. What is binary translation? When and why is it needed?

\filbreak
\vskip 1cm

20. Explain the tradeoffs between static and dynamic binary translation.


\filbreak
\vskip 1cm

21. What is emulation? How is it related to interpretation and simulation.

\filbreak
\vskip 1cm

22. What is dynamic optimization? How can it improve on static optimization?

\filbreak
\vskip 1cm

23. What is binary rewriting? How does it differ from binary translation and dynamic optimization.

\filbreak
\vskip 1cm

24. Describe the notion of a partial execution trace. Why is it important to dynamic optimization and rewriting?

\filbreak
\vskip 1cm

25. What is mobile code?

\filbreak
\vskip 1cm

26. What is sandboxing? When and why is it needed? How can it be implemented?

\filbreak
\vskip 1cm

27. What is reflection? What purpose does it serve?

\filbreak
\vskip 1cm

28. Describe an inappropriate use of reflection.

\filbreak
\vskip 1cm

29. Name an aspect of reflection supported by the CLI but not by the JVM.

\filbreak
\vskip 1cm


30. Why is reflection more difficult to implement in Java or C$\#$ than it is in Perl or Ruby?

\filbreak
\vskip 1cm

31. What are annotations (Java) or attributes (C$\#$)? What are they used for?

\filbreak
\vskip 1cm

32. What are {\tt javadoc}, {\tt apt}, {\tt JML}, and {\tt LINQ}, and what do they have to do with annotation?

\filbreak
\vskip 1cm

33. Briefly describe three different implementation strategies for a symbolic debugger.

\filbreak
\vskip 1cm

34. Explain the difference between breakpoints and watchpoints. Why are watchpoints potentially more expensive?

\filbreak
\vskip 1cm

35. Summarize the capabilities provided by the Unix {\tt ptrace} mechanism.

\filbreak
\vskip 1cm

36. What is the principal difference between the Unix {\tt prof} and {\tt gprof} tools?

\filbreak
\vskip 1cm

37. For the purposes of performance analysis, summarize the relative strengths and limitations of statistical sampling, instrumentation, and hardware performance counters. Explain why statistical sampling and instrumentation might profitably be used together.

\filbreak
\vskip 1cm

38. Summarize the architecture of the Common Language Infrastructure. Constrast it with the JVM. Highlight those features intended to facilitae cross-language interoperability.

\filbreak
\vskip 1cm

39. Describe how the choice of just-in-time compilation (and the rejection of interpretation) influenced the structure of the CLI.

\filbreak
\vskip 1cm

40. Describe several different kinds of references supported by the CLI. Why are there so many?

\filbreak
\vskip 1cm

41. What is the purpose of the Common Language Specification? Why is it only a subset of the Common Type System?

\filbreak
\vskip 1cm

42. Describe the CLI's support for unsafe code. How can this support be reconciled with the need for saftey in embedded settings?
\filbreak
\vfill\eject
\bye
