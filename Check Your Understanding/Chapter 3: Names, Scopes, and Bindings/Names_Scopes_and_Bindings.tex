\centerline{\bf Names, Scopes, and Bindings}

\vskip 1cm

1. What is binding time?

\filbreak
\vskip 1cm

2. Explain the distinction between decisions that are bound statically and those that are bound dynamically.

\filbreak
\vskip 1cm

3. What is the advantage of binding things as early as possible? What is the advantage of delaying bindings?

\filbreak
\vskip 1cm

4. Explain the distinction between the lifetime of a name-to-object binding and its visibility.

\filbreak
\vskip 1cm

5. What determines whether an object is allocated statically, on the stack, or in the heap?

\filbreak
\vskip 1cm

6. List the objects and information commonly found in a stack frame.

\filbreak
\vskip 1cm

7. What is a frame pointer? What is it used for?

\filbreak
\vskip 1cm

8. What is a calling sequence?

\filbreak
\vskip 1cm

9. What are internal and external fragmentation?

\filbreak
\vskip 1cm

10. What is garbage collection?

\filbreak
\vskip 1cm

11. What is a dangling reference?

\filbreak
\vskip 1cm

12. What do we mean by the scope of a name-to-object binding?

\filbreak
\vskip 1cm

13. Describe the difference between static and dynamic scoping.

\filbreak
\vskip 1cm

14. What is elaboration?

\filbreak
\vskip 1cm

15. What is a referencing environment?

\filbreak
\vskip 1cm

16. Explain the closest nested scope rule.

\filbreak
\vskip 1cm

17. What is the purpose of a scope resolution operator?

\filbreak
\vskip 1cm

18. What is a static chain? What is it used for?

\filbreak
\vskip 1cm

19. What are forward references? Why are they prohibited or restricted in many programming languages?

\filbreak
\vskip 1cm

20. Explain the difference between a declaration and a definition. Why is the distinction important?

\filbreak
\vskip 1cm

21. Explain the importance of information hiding.

\filbreak
\vskip 1cm

22. What is an opaque export?

\filbreak
\vskip 1cm

23. Why might it be useful to distinguish between the header and the body of a module?

\filbreak
\vskip 1cm

24. What does it mean for a scope to be closed?

\filbreak
\vskip 1cm

25. Explain the distinction between "modules as managers" and "modules as types."

\filbreak
\vskip 1cm

26. How do classes differ from modules?

\filbreak
\vskip 1cm

27. Why might it be useful to have modules and classes in the same language?

\filbreak
\vskip 1cm

28. Why does the use of dynamic scoping imply the need for run-time type checking?

\filbreak
\vskip 1cm

29. Explain the purpose of a compiler's symbol table.

\filbreak
\vskip 1cm

30. What are aliases? why are they considered a problem in language design and implementation?

\filbreak
\vskip 1cm

31. Explain the value of the {\tt restrict}  qualifier in C.

\filbreak
\vskip 1cm

32. What is overloading? How does it differ from coercion and polymorphism?

\filbreak
\vskip 1cm

33. What are type classes in Haskell? What purpose do they serve?

\filbreak
\vskip 1cm

34. Descibe the difference between deep and shallow binding of referencing environments.

\filbreak
\vskip 1cm

35. Why are binding rules particlarly important for langauges with dynamic scoping?

\filbreak
\vskip 1cm

36. What are first-class subroutines? what languages support them?

\filbreak
\vskip 1cm

37. What is a subroutine closure? What is it used for? How is it implemented?

\filbreak
\vskip 1cm

38. What is an object closure? How is it relatedto a subroutine closure?

\filbreak
\vskip 1cm

39. Describe how the delegates of C$\#$ extend and unify both subroutine and object closures.

\filbreak
\vskip 1cm

40. Explain the distinction between limited and unlimited extent of objects in a local scope.

\filbreak
\vskip 1cm

41. What is a lambda expression? How does the support for lambda expressions in functional languages compare to that of C$\#$ or Ruby? To that of C++ or Java?

\filbreak
\vskip 1cm

42. What are macros? What was the motivation for including them in C? What problems may they cause?

\filbreak
\vskip 1cm

43. List the basic operations provided by a symbol table.

\filbreak
\vskip 1cm

44. Outline the implementation of a LeBlanc-Cook style symbol table.

\filbreak
\vskip 1cm

45. Why don't compilers generally remove names from the symbol table at the ends of their scopes?

\filbreak
\vskip 1cm

46. Describe the association list (A-list) and central reference table data structures used to implement dynamic scoping. Summarize the tradeoffs between them.

\filbreak
\vskip 1cm

47. Explain how to implement deep binding by capturing the referencing environment A-list in a closure. Why are closures harder to build with a central reference table?

\filbreak
\vskip 1cm

48. What purpose(s) does separate compilation serve?

\filbreak
\vskip 1cm

49. What does it mean for an external variable to be linked in C?

\filbreak
\vskip 1cm

50. Summarize the C convention for use of {\tt .h} and {\tt .c} files.

\filbreak
\vskip 1cm

51. Describe the difference between a compilation unit and a C++ or C$\#$ namespace.

\filbreak
\vskip 1cm

52. Explain why Ada and similar languages separate the header of a module from its body. Explain how Java and C$\#$ get by without.
\filbreak
\vfill\eject
\bye
