\centerline{\bf Composite Types}

\vskip 1cm

1. What are struct tags in C? How are they related to type names? How did they change in C++?

\filbreak
\vskip 1cm

2. How do the records of ML differ from those of most other languages?

\filbreak
\vskip 1cm

3. Discuss the significance of "holes" in records. Why do they arise? What problems do they cause?

\filbreak
\vskip 1cm

4. Why is it easier to implement assignment than comparison for records?

\filbreak
\vskip 1cm

5. What is packing? What are its advantages and disadvantages?

\filbreak
\vskip 1cm

6. Why might a compiler reorder the fields of a record? What problems might this cause?

\filbreak
\vskip 1cm

7. Briefly describe two purposes for unions/variant records.

\filbreak
\vskip 1cm

8. What is an array slice? For what purpose are slices useful?

\filbreak
\vskip 1cm

9. Is there any significant difference between a two-dimensional array and an array of one-dimensional arrays?

\filbreak
\vskip 1cm

10. What is the shape of an array?

\filbreak
\vskip 1cm

11. What is a dope vector? What purpose does it serve?

\filbreak
\vskip 1cm

12. Under whatcircumstances can an array declared within a subroutine be allocated in the stack? Under what circumstances must it be allocated in the heap?

\filbreak
\vskip 1cm

13. What is a conformant array?

\filbreak
\vskip 1cm

14. Discuss the comparative advantages of contiguous and row-pointer layout for arrays.

\filbreak
\vskip 1cm

15. Explain the difference between row-major and column-major layout for contiguously allocated arrays. Why does a programmer need to know which layout the compiler uses? Why doo mosy languages designers consider row-major layout to be better?

\filbreak
\vskip 1cm

16. How much of the work of computing the address of an element of an array can be performed at compile time? How much must be performed at run time?

\filbreak
\vskip 1cm

17. Name three languages that provide particularly extensive support for character strings.

\filbreak
\vskip 1cm

18. Why might a language permit operations on strings that it does not provide for arrays?

\filbreak
\vskip 1cm

19. What are the strengths and weaknesses of the bit-vector representation for sets? How else might sets be implemented?

\filbreak
\vskip 1cm

20. Discuss the tradeoffs between pointers and the recursive types that arise naturally in a language with a reference model of variables.

\filbreak
\vskip 1cm

21. Summarise the ways in which one dereferences a pointer in various programming languages.

\filbreak
\vskip 1cm

22. What is the difference between a pointer and an address? Betwen a pointer and a reference?

\filbreak
\vskip 1cm

23. Discuss the advantages and disadvantages of the interoperability of pointers and arrays in C.

\filbreak
\vskip 1cm

24. Under what circumstances must the bounds of a C array be specified in its declaration?

\filbreak
\vskip 1cm

25. What are dangling references? How are they created, and why are they a problem?

\filbreak
\vskip 1cm

26. What is garbage? How is it created, and why is it a problem? Discuss the comparative advantages of reference counts and tracing collection as a means of solving the problem.

\filbreak
\vskip 1cm

27. What are smart pointers? What purpose do they serve?

\filbreak
\vskip 1cm

28. Summarize the differences among mark-and-sweep, stop-and-copy, and generational garbage collection.

\filbreak
\vskip 1cm

29. What is pointer reversal? What problem does it address?

\filbreak
\vskip 1cm

30. What is "conservative" garbage collection? How does it work?

\filbreak
\vskip 1cm

31. Do danging references and garbage ever arise in the same programming language? Why or why not?

\filbreak
\vskip 1cm

32. Why was automatic garbage collection so slow to be adopted by imperative programming languages?

\filbreak
\vskip 1cm

33. What are the advantages and disadvantages of allowing pointers to refer to objects that do not lie in the heap?

\filbreak
\vskip 1cm

34. Why are lists so heavily used in functional programming languages?

\filbreak
\vskip 1cm

35. What are list comprehensions? What languages support them?

\filbreak
\vskip 1cm

36. Compare and contrast the support for lists in ML- and Lisp-family laguages.

\filbreak
\vskip 1cm

37. Explain the distinction between interactive and file-based {\tt I/O}; between temporary and persistent files.

\filbreak
\vskip 1cm

38. What are some of the tradeoffs between supporting {\tt I/O} in the language proper versus supporting it in libraries?

\filbreak
\vskip 1cm

39. What are anonymous unions and structs? What purpose do they serve? How is this related to the integration of variants with records in Pascale and its descendants?

\filbreak
\vskip 1cm

40. What is a tag (discriminant) in a variant record? In a language like Ada or OCaml, how does it differ from an ordinary field?

\filbreak
\vskip 1cm

41. Discuss the type safety problems that arise with variant records. How can these problems be addressed?

\filbreak
\vskip 1cm

42. Summarize the rules that prevent access to inappropriate fields of variant records in OCaml and Ada.

\filbreak
\vskip 1cm

43. Why might one wish to constrain a variable, so that it can hold only one variant of a type?

\filbreak
\vskip 1cm

44. Explain how classes and inheritance can be used to obtain the effect of constrained variant records.

\filbreak
\vskip 1cm

45. What are tombstones? What changes do they require in the code to allocate and deallocate memory, and to assign and dereference pointers?

\filbreak
\vskip 1cm

46. Explain how tombstones can be used to support compaction.

\filbreak
\vskip 1cm

47. What are locks and keys? What changes do they require in the code to allocate and deallocate memory, and to assign and dereference pointers?

\filbreak
\vskip 1cm

48. Explain why the protection afforded by locks and keys is only probabilistic.

\filbreak
\vskip 1cm

49. Discuss the comparative advantages of tombstones and locks and keys as a means of catching dangling references.

\filbreak
\vskip 1cm

50. Explain the differences between interactive and file-based {\tt I/O}, between temporary and persistent files, and between binary and text files.

\filbreak
\vskip 1cm

51. What are the comparative advantages of text and binary files?

\filbreak
\vskip 1cm

52. Describe the end-of-line conventions of Unix, Windows, and Macintosh files.

\filbreak
\vskip 1cm

53. What are the advantages and disadvantages of building {\tt I/O} into a programming language, as opposed to providing it through a library routines?

\filbreak
\vskip 1cm

54. Summarize the different approaches to text {\tt I/O} adopted by Fortran, Ada, C, and C++.

\filbreak
\vskip 1cm

55. Describe some of the weaknessess of C's {\tt scanf} mechanism.

\filbreak
\vskip 1cm

56. What are stream manipulators? How are they used in C++?

\filbreak
\vfill\eject
\bye
