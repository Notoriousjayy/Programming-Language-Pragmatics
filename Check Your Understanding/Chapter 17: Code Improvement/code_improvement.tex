\centerline{\bf Code Improvement}

\vskip 1cm

1. Describe several levels of "aggressiveness" in code improvement.

\filbreak
\vskip 1cm

2. Give three examples of code improvements that must be performed in a particular order. Give two examples of code improvements that should probably be performed more than once (with other improvements in between).

\filbreak
\vskip 1cm

3. What is a peephole optimization? Describe at least four different ways in which a peephole optimizer might transform a program.

\filbreak
\vskip 1cm

4. What is a constant folding? Constant propagation? Copy propagation? Strength reduction?

\filbreak
\vskip 1cm

5. What does it mean for a value in a register to be live?

\filbreak
\vskip 1cm

6. What is a control flow graph? Why is it central to so many forms of global code improvement? How does it accomodate subroutine calls?

\filbreak
\vskip 1cm

7. What is value numbering? What purpose does it serve?

\filbreak
\vskip 1cm

8. Explain the connection between common subexpressions and expression rearrangement.

\filbreak
\vskip 1cm

9. Why is it not practical in general for the programmer to eliminate common subexpressions a the source level?

\filbreak
\vskip 1cm

10. What is static single assignment (SSA) form? Why is SSA form needed for global value numbering, but not for local value numbering?

\filbreak
\vskip 1cm

11. What are merge functions in context of SSA form?

\filbreak
\vskip 1cm

12. Give three distinct examples of data flow analysis. Explain the difference between forward and backward flow. Explain the difference between all-paths and any-path flow.

\filbreak
\vskip 1cm

13. Explain the role of the In, Out, Gen, and Kill sets common to many examples of data flow analysis.

\filbreak
\vskip 1cm

14. What is a partially redundant computation? Why might an algorithm to eliminate partial redundancies need to split an edge in a control flow graph?

\filbreak
\vskip 1cm

15. What is an available expression?

\filbreak
\vskip 1cm

16. What is forward substitution?

\filbreak
\vskip 1cm

17. What is live variable analysis? What purpose does it serve?

\filbreak
\vskip 1cm

18. Describe at least three instances in which code improvement algorithms must consider the possibility of aliases.

\filbreak
\vskip 1cm

19. What is a loop invariant? A reaching definition?

\filbreak
\vskip 1cm

20. Why might it sometimes be unsafe to hoist an invariant out of a loop?

\filbreak
\vskip 1cm

21. What are induction variables? What is strength reduction?

\filbreak
\vskip 1cm

22. What is control flow analysis? Why is it less important than it used to be?

\filbreak
\vskip 1cm

23. What isregister pressure? Register spilling?

\filbreak
\vskip 1cm

24. Is instruction scheduling a machine-independent code improvement technique? Explain.

\filbreak
\vskip 1cm

25. Describe the creation and use of a dependence DAG. Explain the distinctions among flow, anti-, and output dependences.

\filbreak
\vskip 1cm

26. Explain the tension between instruction scheduling and register allocation.

\filbreak
\vskip 1cm

27. List several heuristics that might be used to prioeritize instructions to be scehduled.

\filbreak
\vskip 1cm

28. What is the difference between loop unrolling and software pipelining? Explain why the latter may increase register pressure.

\filbreak
\vskip 1cm

29. What is the purpose of loop interchange? Loop tiling (blocking)?

\filbreak
\vskip 1cm

30. What are the potential benefits of loop distribution? Loop fusion? What is loop peeling?

\filbreak
\vskip 1cm

31. What does it mean for loops to be perfectly nested? Why are perfect nests important?

\filbreak
\vskip 1cm

32. What is a loop-carried dependence? Describe three loop transformations that may serve in some cases to eliminate such a dependence.

\filbreak
\vskip 1cm

33. Describe the fundamental difference between the parellelization strategy for multicore machines and the parallelization strategy for vector machines.

\filbreak
\vskip 1cm

34. What is self scheduling? When is it desirable?

\filbreak
\vskip 1cm

35. What is the live range of a register? Why might it not be a contiguous range of instructions?

\filbreak
\vskip 1cm

36. What is a register interference graph? What is its significance? Why do production compilers depend on heuristics (rather than precise solutions) for register allocation?

\filbreak
\vskip 1cm

37. List three reasons why it might not be possible to treat architectural registers uniformaly for purposes of register allocation.


\filbreak
\vfill\eject
\bye
