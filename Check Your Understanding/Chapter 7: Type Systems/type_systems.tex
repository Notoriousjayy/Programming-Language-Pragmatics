\centerline{\bf Type Systems}

\vskip 1cm

1. What purpose(s) do types serve in a programming language?

\filbreak
\vskip 1cm

2. What does it mean for a language to be strongly typed? Statically typed? What prevents, say, C from being strongly typed?

\filbreak
\vskip 1cm

3. Name two programming laguages that are strongly but dynamically typed.

\filbreak
\vskip 1cm

4. What is a type clash?

\filbreak
\vskip 1cm

5. Discuss the differences among the denotational, structural, and abstraction-based views of types.

\filbreak
\vskip 1cm

6. What does it mean for a set of language features (e.g., a type system) to be orthogonal?

\filbreak
\vskip 1cm

7. What are aggregates?

\filbreak
\vskip 1cm

8. What are option types? What purpose do they serve?

\filbreak
\vskip 1cm

9. What is polymorphism? What distinguishes its parametric and subtype varieties? What are generics?

\filbreak
\vskip 1cm

10. What is the difference between discrete and scalar types?

\filbreak
\vskip 1cm

11. Give two examples of languages that lack a Boolean type. What do they use instead?

\filbreak
\vskip 1cm

12. In what ways may an enumeration type be preferable to a collection of named constants? In what ways may a subrange type be preferable to its base type? In what ways may a string be preferable to an array of characters?

\filbreak
\vskip 1cm

13. What is the difference between type equivalence and type compatibility?

\filbreak
\vskip 1cm

14. Discuss the comparative advantages of structural and name equivalence for types. Name three languages that use each approach.

\filbreak
\vskip 1cm

15. Explain the difference between strict and loose name equivalence.

\filbreak
\vskip 1cm

16. Explain the distinction between derived types and subtypes in Ada.

\filbreak
\vskip 1cm

17. Explain the differences among type conversion, type coercion, and nonconverting type casts.

\filbreak
\vskip 1cm

18. Summarize the arguments for and against coercion.

\filbreak
\vskip 1cm

19. Under what circumstances does a type conversion require a run-time check?

\filbreak
\vskip 1cm

20. What purpose is served by universal reference types?

\filbreak
\vskip 1cm

21. What is a type inference? Describe three contexts in which it occurs.

\filbreak
\vskip 1cm

22. Under what circumstances does an ML compiler announce a type clash?

\filbreak
\vskip 1cm

23. Explain how the type inference of ML leads naturally to polymorphism.

\filbreak
\vskip 1cm

24. Why do ML programmers often declare the types of variables, even when they don't have to?

\filbreak
\vskip 1cm

25. What is unification? What is its role in ML?

\filbreak
\vskip 1cm

26. Explain the distinction between implicit and explicit parametric polymorphism. What are their comparative advantages?

\filbreak
\vskip 1cm

27.  What is duck typing? What is its connection to polymorphism? In what languages does it appear?

\filbreak
\vskip 1cm

28. Explain the distinction between overloading and generics. Why is the former sometimes called ad hoc polymorphism?

\filbreak
\vskip 1cm

29. What is the principal purpose of generics? In what sense do generics serve to a broader purpose in C++ and Ada than they dp in Java and C$\#$?

\filbreak
\vskip 1cm

30. Under what circumstances can a language implementation share code amonf separate instances of a generic?

\filbreak
\vskip 1cm

31. What are container classes? What do they have to do with generics?

\filbreak
\vskip 1cm

32. What does it mean for a generic parameter to be constrained? Explain the difference between explicity and implicit constraints. Describe how interface classes can be used to specify constraints in Java and C$\#$/

\filbreak
\vskip 1cm

33. Why will C$\#$ accept {\tt int} as a generic argument, but Java won't?

\filbreak
\vskip 1cm

34. Under what circumstances will C++ instantiate a generic function implicitly?

\filbreak
\vskip 1cm

35. Why is equality testing more subtle than it first appears?

\filbreak
\vskip 1cm

36. Why is it difficult to produce high-quality error messages for misuses of C++ templates?

\filbreak
\vskip 1cm

37. What is the purpose of explicit instantiation in C++? What is the purpose of {\tt extern} templates?

\filbreak
\vskip 1cm

38. What is template metaprogramming?

\filbreak
\vskip 1cm

39. Explain the difference between upper bounds and lower bounds in Java type constraints. Which of these does C$\#$ support?

\filbreak
\vskip 1cm

40. What is typeerasure? Why is it used in Java?

\filbreak
\vskip 1cm

41. Under what circumstances will a Java compiler issue an "unchecked" generic warning?

\filbreak
\vskip 1cm

42. Why must fields of generic parameter type be explicitly initialized in C$\#$?

\filbreak
\vskip 1cm

43. For what two main reasons are C$\#$ generics often more efficient than comparable code in Java?

\filbreak
\vskip 1cm

44. Summarize the notations of covariance and contravariance in generic types.

\filbreak
\vskip 1cm

45. Hoes does a C$\#$ delegate differ from an interface with a single method (e.g., the C++ chooser of Figure c-7.5)? How does it differ from a function pointer in C?

\filbreak
\vfill\eject
\bye
