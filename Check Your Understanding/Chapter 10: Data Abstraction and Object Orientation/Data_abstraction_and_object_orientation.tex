\centerline{\bf Data abstraction and Object Orientation}

\vskip 1cm

1. What are generally considered to be the three defining characteristics of object-oriented programming?

\filbreak
\vskip 1cm

2. In what programming language of the 1960s does object orientation find its roots? Who invented that language? Summarize the evolution of the three defining characteristics since that time.

\filbreak
\vskip 1cm

3. Name three important benefits of abstraction.

\filbreak
\vskip 1cm

4. What are the more common names for subroutine member and data memebers?

\filbreak
\vskip 1cm

5. What is a property in C$\#$.

\filbreak
\vskip 1cm

6. What is the purpose of the "private" part of an object interface? Why can't it be hidden completely?

\filbreak
\vskip 1cm

7. What is the purpose of the {\tt ::} operator in C++?

\filbreak
\vskip 1cm

8. Explain why in-line subroutines are particularly important in object-oriented languages.

\filbreak
\vskip 1cm

9. What are constructors and destructors?

\filbreak
\vskip 1cm

10. Give two other terms, each, for base class and derived class.

\filbreak
\vskip 1cm

11. Explain why generics may be useful in an object-oriented language, despite the extensive polymorphism already provided by inheritance.

\filbreak
\vskip 1cm

12. What is meant by an opaque export from a module?

\filbreak
\vskip 1cm

13. What are private types in Ada?

\filbreak
\vskip 1cm

14. Explain the significance of the {\tt this} parameter in object-oriented languages.

\filbreak
\vskip 1cm

15. How do Java and C$\#$ make do without explicit class headers?

\filbreak
\vskip 1cm

16. Explain the distinctions among {\tt private}, {\tt protected}, and {\tt public} class members in C++.

\filbreak
\vskip 1cm

17. Explain the distinctions among {\tt private}, {\tt protected}, and {\tt public} base class members in C++.

\filbreak
\vskip 1cm

18. Describe the notion of selective availability in Eiffel.

\filbreak
\vskip 1cm

19.  How do the rules for member name visibility in Smalltalk and Objective-C differ from the rules of most object-oriented languages?

\filbreak
\vskip 1cm

20. Howdo inner classes in Java differ from most nested classes?

\filbreak
\vskip 1cm

21. Describe the key design difference between object-oriented features of Smalltalk, Eiffel, and C++, on the one hand, and Ada, CLOS, and Fortran on the other.

\filbreak
\vskip 1cm

22. What are extension methods in C$\#$? What purpose do they serve?

\filbreak
\vskip 1cm

23. Does a constructor allocate space for an object? Explain.

\filbreak
\vskip 1cm

24. What is a metaclass in Smalltalk?

\filbreak
\vskip 1cm
 
25. Why is object initialization simpler in a language with a reference model of variables (as opposed t a value model)?

\filbreak
\vskip 1cm

26. How does a C++ (or Java or C$\#$) compiler tell which constructor to use for a given object? How does the answer differ for Eiffel and Smalltalk?

\filbreak
\vskip 1cm

27. What is escape analysis? Describe why it might be useful in a language with a reference model of variables.

\filbreak
\vskip 1cm

28. Summarize the rules in C++ that determine the order in which constructors are called for a class, its base class(es), and the classes of its fields. How are these rules simplified in other languages?

\filbreak
\vskip 1cm

29. Explain the difference between initialization and assignment in C++.

\filbreak
\vskip 1cm

30. Why does C++ need destructors more than Eiffel does?

\filbreak
\vskip 1cm

31. Explain the difference between dynamic and static method binding (i.e., between virtual and nonvirtual methods).

\filbreak
\vskip 1cm

32. Summarize the fundamental argument for dynamic method binding. Why do C++ and C$\#$ use static binding by default?

\filbreak
\vskip 1cm

33. Explain the distinction between redefining and overriding a method.

\filbreak
\vskip 1cm

34. What is a class-wide type in Ada 95?

\filbreak
\vskip 1cm

35. Explain the connection between dynamic method binding and polymorphism.

\filbreak
\vskip 1cm

36. What is an abstract method (also called a pure virtual method) in C++ and a deferred feature in Eiffel)?


\filbreak
\vskip 1cm

37. What is a reverse assignment? Why does it require a run-time check?

\filbreak
\vskip 1cm

38. What is a vtable? How is it used?

\filbreak
\vskip 1cm

39. What is the fragile base class problem?

\filbreak
\vskip 1cm

40. What is an abstract (deferred) class?

\filbreak
\vskip 1cm

41. Explain the importance of virtual methods for object closures.

\filbreak
\vskip 1cm

42. What is mix-in inheritance? What problem does it solve?

\filbreak
\vskip 1cm

43. Outline a possible implementation of mix-in inheritance for a language with statically typed objects. Explain in particular the need for interface-specific views of an object.

\filbreak
\vskip 1cm

44. Describe how mix-ins (and their implementation) can be extended with default method implementations, static (constant) fields, and even mutable fields.

\filbreak
\vskip 1cm

45. What does true multiple inheritance make possible that mix-in inheritance does not?

\filbreak
\vskip 1cm

46. What is repeated inheritance? What is the distinction between replicated and shared repeated inheritance?

\filbreak
\vskip 1cm

47. What does it mean for a language to provide a uniform object mode? Name two languages that do so.

\filbreak
\vskip 1cm

48. Given a few examples of the semantic ambiguities that arise when a class has more than one base class.

\filbreak
\vskip 1cm

49. Explain the distinction between replicated and shared multiple inheritance. When is each desirable?

\filbreak
\vskip 1cm

50. Explain how even nonrepeated multiple inheritance introduces the need for "{\tt this} correction" fields in individual vtable entries.

\filbreak
\vskip 1cm

51. Explain how shared multiple inheritance introduces the need for an additional level of indirection when accessing fields of certain parent classes.

\filbreak
\vskip 1cm

52. Explain why true multiple inheritance is harder to implement than mix-in inheritance.

\filbreak
\vskip 1cm

53. Name the three projects at Xerox PARC in the 1970s that pioneered modern GUI-based personal computers.

\filbreak
\vskip 1cm

54. Explain the concept of a message in Smalltalk.

\filbreak
\vskip 1cm

55. How does Smalltalk indicate multiple message arguments?

\filbreak
\vskip 1cm

56. What is a block in Smalltalk? What mechanism does it resemble in Lisp?

\filbreak
\vskip 1cm

57. Give three examples of how Smalltalk models control flow as message evaluation.

\filbreak
\vskip 1cm

58. Explain how type checking works in Smalltalk.

\filbreak
\vfill\eject
\bye
