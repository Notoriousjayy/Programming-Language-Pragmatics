Java supports three jump statements {\bf break, continue}, and {\bf return}.
\vskip 1cm
{\bf break}
\vskip 3mm
The {\bf break} statement has three uses. First, it terminates a statement sequence in a {\bf switch} statement. Second, it can be used to exit a loop. Third, it can be used as a structured form of {\bf goto}.
\vskip 1mm
By using {\bf break}, you can force immediate termination of a loop, bypassing the conditional expression and any remaining code in the body of the loop. When a {\bf break} statement is encountered inside a loop, the loop is terminated and program control resumes at the next statement.

\vskip 1mm
When used inside a set of nested loops, the {\bf break} statement will only break out of the innermost loop.

\vskip 1mm
Here are two important points to remember about {\bf break}. First, more than one {\bf break} statement may appear in a loop. Be careful. Too many {\bf break} statements can destructure your code. Second, the {\bf break} that terminates a {\bf switch} statement affects only that {\bf switch} statement and not any enclosing loops.

\vskip 1in
\filbreak
{\bf Using break as a Form of Goto}
\vskip 3mm
The {\bf break} statement can also be used as a structured form of the {\bf goto} statement. Java does not have a goto statement because it provides a way to branch in an arbitrary and unstructured way. There are places where the goto is a valuable and legitimate construcut for flow control. The goto can be useful when you are exiting from a deeply nested set of loops. To handle such situations, Java defines an expanded form of the {\bf break} statement. By using {\bf break} you can break out of one or more blocks. These blocks don't need to be part of a loop or a {\bf switch}. They can be any block. You can specify precisely where execution will resume, because this form of {\bf break} works with a label.

\vskip 1mm
Here is the general form of the labeled {\bf break} statement:

$$\hbox{\tt break {\it label};}$$

\vskip 1mm
Most often, {\it label} is the name of a label that identifies a block of code. Than be a standalone block of code but it can also be a block that is the target of another statement, When this form of {\bf break} executes, control is transferred out of the named block. The labeled block must enclose the {\bf break} statement, but it does not need to be the immediately enclosing block. This means, for example, that you can use a labeled {\bf break} statement to exit from a set of nested blocks. But you cannot use {\bf break} to transfer control out of a block that does not enclose the {\bf break} statement.

\vskip 1mm
To name a block, put a label at the start of it. A {\it label} is any valid Java identifier followed by a colon. Once you have labeled a block, you can then use this label as the target of a {\bf break} statement. Doing so causes execution to resume at the {\it end} of the labeled block.

\filbreak
\vfill\eject
\bye
