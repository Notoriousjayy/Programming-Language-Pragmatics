\nopagenumbers
{\bf The Binding of Reference Environments}
\vskip 1mm
\hrule

\vskip 6pt
{\bf Shallow binding}: Late binding of the referencing environment of a subroutine that has been passed as a parameter. Is usually the default in languages with {\bf dynamic scoping}.

\vskip 6pt
{\bf Deep binding}: Early binding of the referencing environment of a subroutine that has been passed as a parameter. is almost always default in languages with {\bf static scoping}.

\vskip 6pt
Static scope rules specify that the referencing environment depends on the lexical nesting of program blocks in which names are declared. Dynamic scope rules specify that the referencing environment depends on the order in which declarations are encountered at run time. In languages that allow one to create {\it reference} to a subroutine---by passing it as a parameter. When should scope rules be applied to such a subroutine: when the reference is first created, or when the routine is finally called?

\vfill\eject
\bye
