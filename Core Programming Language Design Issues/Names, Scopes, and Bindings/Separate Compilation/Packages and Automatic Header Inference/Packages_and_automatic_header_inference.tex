\nopagenumbers
{\bf Packages and Automatic Header Inference}
\vskip 1mm
\hrule

\vskip 6pt
The separate compilation facilities of Java and C$\#$ eliminate {\tt .h} files. Java introduces a formal notion of module, called a {\bf package}. Every compilation unit, which may be a file or (in some implementations) a record in a database, belongs to exactly one package, but a package may consists of many compilation units, each of which  begins with an indication of the package it belongs to. Unless explicitly declared as {\tt public}, a class in Java is visible in all and only those compilation units that belong to the same package.

\vskip 6pt
As in C++, a compilation unit that needs to use classes from another package can access them using qualified names, or via name-at-time or package-at-a-time import.

\vskip 6pt
When asked to import names from package $M$, the Java compiler will search for $M$ in a standard set of places, and will recompile it if appropriate. The compiler will then {\bf automatically} extract the information that would have been needed in a C++ {\tt .h} file or an Ada or Modula-3 header. If the compilation of $M$ requires other packages, the compiler will search for them as well, recursively.

\vskip 6pt
To mimic the software engineering practice of early header file construction, a Java or C$\#$ design team can create a skeleton version of (the public classes of) its packages or namespaces, which can then be used, concurrently and independently, by the programmers responsible for the full versions.

\vfill\eject
\bye
