\nopagenumbers
{\bf Names, Scopes, and Bindings}
\vskip 1mm
\hrule

\vskip 6pt
{\bf High-Level Language (HLL)}: The language syntax and semantics are significantly more abstract--- farther from the hardware.

\vskip 6pt
{\bf Name}: a name is a mnemonic character string used to represent something else. Names in most languages are $\underline{\hbox{identifiers}}$ (alphanumeric tokens). Names allow us to refer to {\bf variables, constants, operations,} and {\bf types} using $\underline{\hbox{symbolic identifiers}}$ rather than low-level concepts like $\underline{\hbox{addresses}}$

\vskip 6pt
{\bf Abstraction}: a process by which the programmer associates a name with a potentially complicated program fragment, which can them be thought of in terms of its purpose or function, rather than in terms of how that function is achieved.

\vskip 6pt
{\bf Subroutines} are $\underline{\hbox{control abstractions}}$: they allow the programmer to hide arbitrarily complicated code behind a simple interface.

\vskip 6pt
{\bf Classes} are $\underline{\hbox{data abstractions}}$: they allow a programmer to hide data representation details behind a simple set of operations.

\vskip 6pt
{\bf Referencing Environment}: The complete set of bindings in effect at a given point in a program.

\vskip 6pt
{\bf Macro Expansion}: introduces new names via textual substitution.

\vskip 6pt
Binding time refers not only to the binding of a name to the thing it represents, but also in general to the notion of resolbing any design decision in a language implementation.


\vfill\eject
\bye
