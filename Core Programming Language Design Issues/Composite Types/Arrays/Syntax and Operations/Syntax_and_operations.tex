\nopagenumbers
{\bf Syntax and Operations}
\vskip 1mm
\hrule

\vskip 6pt
Most languages refer to an elemen of an array by appending a subscript---usualy delimited by square brackets---to the name of the array: {\tt Arrray$\lbrack 3 \rbrack$}

\vskip 6pt
In some languages one declares an array by appending subscript notation to the syntax that would be used to declare a scalar. In C:

$$\hbox{{\tt char upper$\lbrack N \rbrack$;}}$$

In C, the lower bound of an index range is always zero: the indices of an $n$-element array are $0\ldots n-1$.

\vskip 6pt
In some languages, one can also declare a {\bf multi-dimensional} array (array of arrays) by using the {\tt array} constructor more than once in the same declaration.

\vskip 6pt
In C, one must also declare an array of arrays, and use two-subscript notation, but C's integration of pointers and arrays means that slices are not supported.

$$\hbox{{\tt char matrix$\lbrack N \rbrack$$\lbrack N \rbrack$;}}$$

given this definition, {\tt matrix$\lbrack 3\rbrack \lbrack 4\rbrack$} denotes an individual element of the array, but {\tt matrix$\lbrack 3 \rbrack$} denotes a {\bf reference}, to either the third row of the array or to the first  element of that row, depending on context.

\vskip 12pt

{\bf Slices and Array Operations}
\vskip 1mm
\hrule

\vskip 6pt
A {\bf slice} or {\bf section} is a rectangular portion of an array. Many scripting languages provide extensive faciliries for slicing.

\vskip 6pt
In most languages, the only operations permitted on an array are slection of an element, and assignment. A few languages allow arrays to be compared for equality.


\vfill\eject
\bye
